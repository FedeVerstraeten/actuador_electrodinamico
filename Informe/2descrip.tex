\section{Descripción}
\label{sec:descrip}
En este caso particular, dado que se utiliza un esquema interferométrico como el de Michelson, puede no ser necesario contar con transductores tan precisos como los piezoeléctricos, pudiéndose optar por alternativas que resulten más económicas.
De esta manera, en el presente trabajo nos planteamos obtener un método de transducción eléctrica-mecánica a partir de dispositivos de uso común cuyo sistema de control incluya la posibilidad de definir el tipo de desplazamiento que se desea utilizar por medio de Matlab. Particularmente, se buscará generar el desplazamiento de un espejo conectándolo a un parlante o altavoz, el cual será controlado por Matlab mediante la presencia de un microcontrolador como intermediario entre estos.

Se comenzará por analizar el sistema interferométrico utilizado para poder definir los parámetros más significativos de control que se requieren (desplazamiento necesario, aspectos mecánicos y de diseño, opciones de control, etc.). Finalmente, una vez diseñados e implementados, se caracterizarán mecánicamente con la misma instrumentación interferométrica para obtener una evaluación de la performance del sistema completo.